\documentclass[11pt,american,usenames,dvipsnames,svgnames,x11names,table]{article}
\usepackage{geometry}
\geometry{twoside,margin=2.5cm,
columnsep=2.5cm,
paperheight=297mm,paperwidth=210mm}

%\geometry{
%showframe=true,
%showcrop=true,
twoside=false,
twocolumn=true,
landscape=true,
margin=16mm,
columnsep=12mm,
paperheight=297mm,
paperwidth=160mm}
%-------------------------------------------------------------------------------
\usepackage{fancyhdr}
\pagestyle{fancy}
%\setlength{\headwidth}{2 \columnwidth}
%\addtolength{\headwidth}{\columnsep}
\setlength{\headwidth}{\textwidth}
\fancypagestyle{plain}{
\fancyhead{} % clear all header fields 
\fancyfoot{} % clear all foot fields
\fancyfoot[R]{\textsf{\thepage}} 
\fancyfoot[L]{\textsf{Draft of \today}}
\renewcommand{\headrulewidth}{0.0pt}
\renewcommand{\footrulewidth}{0.1pt}}
\pagestyle{plain}
%-------------------------------------------------------------------------------

\usepackage{color}
\usepackage{xcolor}
\definecolor{pblue}{rgb}{0.13,0.13,1}
\definecolor{pgreen}{rgb}{0,0.5,0}
\definecolor{pred}{rgb}{0.9,0,0}
\definecolor{pgrey}{rgb}{0.46,0.45,0.48}
%-------------------------------------------------------------------------------
\usepackage{url}
\usepackage{babel}
%\usepackage[gen]{eurosym}
\usepackage{times}
%-------------------------------------------------------------------------------
\usepackage{amsthm}
\usepackage{amsmath}
\usepackage{amssymb}
\numberwithin{equation}{section}
\numberwithin{figure}{section}
\usepackage{graphics}
\usepackage{epsfig}
\usepackage{graphicx}
\PassOptionsToPackage{normalem}{ulem}
\usepackage{ulem}
\usepackage[numbers]{natbib} 
\DeclareMathOperator*{\argmin}{argmin}
\DeclareMathOperator*{\argmax}{argmax}
%\DeclareMathOperator*{\cdf}{cdf}
%\DeclareMathOperator*{\quantile}{quantile}
\makeatother
%-------------------------------------------------------------------------------
\usepackage{algpseudocode,algorithm,algorithmicx}
%-------------------------------------------------------------------------------
\usepackage[unicode=true,pdfusetitle,
 bookmarks=true,bookmarksnumbered=false,bookmarksopen=true,bookmarksopenlevel=1,
 breaklinks=false,pdfborder={0 0 0},backref=false,colorlinks=true]{hyperref}
\hypersetup{unicode=true,
colorlinks=true,
pdfpagemode=UseOutlines,
pdfpagelayout=OneColumn,
pdfstartview=Fit,
linkcolor=MidnightBlue,
urlcolor=Mahogany,
citecolor=OliveGreen}
\makeatletter
%-------------------------------------------------------------------------------
\usepackage{datetime}
\renewcommand{\dateseparator}{-}
\renewcommand{\today}{
\the\year \dateseparator \twodigit\month \dateseparator \twodigit\day}
%-------------------------------------------------------------------------------
\setlength{\parskip}{3pt}
\setlength{\parindent}{0pt}
\usepackage[parfill]{parskip}
\usepackage{fancyhdr}
\fancypagestyle{plain}{
\fancyhead{} % clear all header fields 
\fancyfoot{} % clear all foot fields
\fancyfoot[RO,LE]{\thepage} 
\fancyfoot[RE,LO]{Draft of \today}
\renewcommand{\headrulewidth}{0.0pt}
\renewcommand{\footrulewidth}{0.1pt}}
\pagestyle{plain}
%-------------------------------------------------------------------------------
\usepackage[T1]{fontenc}
\usepackage{inconsolata}
% \usepackage{xeCJK}
% \setCJKmainfont{SimHei}
% \setCJKsansfont{SimHei}
% \setCJKmonofont{Lucida Sans Typewriter}
\usepackage{fontspec}
%\setmainfont{Baskerville Old Face}
%\setmainfont{Centaur}
%\setmainfont{Garamond}
%\setmainfont{Georgia}
%\setmainfont{Perpetua}
%\setmainfont{Poor Richard}
\setsansfont{Gill Sans MT} 
\newfontfamily\headingfont[]{Gill Sans MT}
%\newfontfamily\headingfont[]{Perpetua Titling MT}
%-------------------------------------------------------------------------------
\usepackage{listings}
\lstset{language=Java,
  %backgroundcolor={\color{GhostWhite}},
  showspaces=false,
  showtabs=false,
  %breaklines=true,
  breaklines=false,
  %showstringspaces=false,
  breakatwhitespace=true,
  commentstyle=\color{pgreen},
  keywordstyle=\color{pblue},
  stringstyle=\color{pred},
  basicstyle=\ttfamily,
  %basicstyle={\ttfamily\small},
  captionpos=b,
  frame=tblr,
  moredelim=[il][\textcolor{pgrey}]{$$},
  moredelim=[is][\textcolor{pgrey}]{\%\%}{\%\%}
}
%\renewcommand{\lstlistingname}{Listing}
%-------------------------------------------------------------------------------
\providecommand{\algorithmname}{Algorithm}
\providecommand{\exercisename}{Exercise}
\providecommand{\theoremname}{Theorem}
%-------------------------------------------------------------------------------
\usepackage{enumitem}
\setlist[description]{font=\sffamily\normalsize\mdseries,style=unboxed,leftmargin=0cm}
\setlist[itemize]{style=unboxed,itemindent=0cm}
\setlist[enumerate]{style=unboxed,itemindent=0cm}
%-------------------------------------------------------------------------------
\usepackage{titling}
\renewcommand{\maketitlehooka}{\headingfont}
\pretitle{\begin{flushright}\Large\sffamily\bfseries}
\posttitle{\par\end{flushright}\vskip 0.25em}
\preauthor{\begin{flushright}}
\postauthor{\par\end{flushright}}
\predate{\begin{flushright}}
\postdate{\par\end{flushright}}
\setlength{\droptitle}{-80pt}
%-------------------------------------------------------------------------------
\usepackage[sf,small,compact]{titlesec}
\setcounter{secnumdepth}{5} 
% \expandafter, etc, to get rid of texlipse warnings about \section
% without arg
\expandafter\titleformat\expandafter{\csname section\endcsname}[hang]
  {\Large\headingfont\bfseries}
  %{\S\thesection}
  {\thesection}
  {0.5em}
  {}
\renewcommand\thesection{\arabic{section}}

\expandafter\titleformat\expandafter{\csname
subsection\endcsname}[hang] {\large\headingfont\bfseries}
  %{\S\thesubsection}
  {\thesubsection}
  {0.5em}
  {}
\renewcommand\thesubsection{\arabic{section}.\arabic{subsection}}

\expandafter\titleformat\expandafter{\csname subsubsection\endcsname}[hang]
  {\normalsize\headingfont\bfseries}
  %{\S\thesubsubsection}
  {\thesubsubsection}
  {0.5em}
  {}
\renewcommand\thesubsubsection{\arabic{section}.\arabic{subsection}.\arabic{subsubsection}}

\expandafter\titleformat\expandafter{\csname
paragraph\endcsname}[runin] {\normalsize\headingfont\bfseries}
  %{\S\theparagraph}
  {\theparagraph}
  {0.5em}
  {}[\hspace{1em}]
\renewcommand\theparagraph{}
%\renewcommand\theparagraph{\arabic{section}.\arabic{subsection}.\arabic{subsubsection}.\arabic{paragraph}}

\expandafter\titleformat\expandafter{\csname
subparagraph\endcsname}[runin] {\normalsize\headingfont\mdseries}
  %{\S\thesubparagraph}
  {\thesubparagraph}
  {0.5em}
  {}[\hspace{1em}]
\renewcommand\thesubparagraph{}
%\renewcommand\thesubparagraph{\arabic{section}.\arabic{subsection}.\arabic{subsubsection}.\arabic{paragraph}.\arabic{subparagraph}}

%\titleformat{\section}[hang]{\thesection}
%\titleformat{\subsection}[hang]{\large\headingfont\mdseries}{\thesubsection}{0.5em}{}
%\titleformat{\subsubsection}[hang]{\large\headingfont\mdseries}{\thesubsubsection}{0.5em}{}
%\titleformat{\paragraph}[runin]{\normalsize\headingfont\mdseries}{\theparagraph}{0.5em}{}[\hspace{1em}]
%\titleformat{\subparagraph}[runin]{\small\headingfont}{\thesubparagraph}{0.5em}{}[\hspace{1em}]
% \titlespacing\section{0pt}{4pt plus 4pt minus 2pt}{0pt plus 2pt minus 2pt} 
% \titlespacing\subsection{0pt}{4pt plus 4pt minus 2pt}{0pt plus 2pt minus 2pt} 
% \titlespacing\subsubsection{0pt}{4pt plus 4pt minus 2pt}{0pt plus 2pt minus 2pt}
%-------------------------------------------------------------------------------

% John Alan McDonald 2011-09-09

\def\F{{\mathbf F}}  
\def\T{{\mathbf T}}  
\def\G{{\mathbf G}}  

\def\prob{\rho}
\def\otherprob{\phi}
\def\uniform{\upsilon}
\def\gaussian{\nu}
\def\Likelihood{\mathcal{L}}
%\def\cost{\kappa}    
\def\score{\mathcal{K}}
\def\cost{\mathcal{C}}
\def\cmplxty{\mathcal{S}}
\def\constraints{\mathbb{K}}
\def\loss{\mathcal{L}}    
\def\risk{\mathcal{R}}    
\def\accumulate#1#2{\overset{#2}{\underset{#1}{\mathcal{S}}}}    
\def\ploss{\varphi}    
\def\asin{\mathrm{ASIN}}
\def\product{\mathrm{product}}
\def\statistic{\xi}    
\def\mean{\text{mean}}    
\def\var{\text{var}}    
\def\median{\text{median}}    

%\def\cdf#1{{\mathrm{cdf}_{#1}}} 
%\def\quantile#1{{\cdf{#1}^{-}}} 
%\def\quantile#1{{Q_{#1}}} 
\def\quantile{\text{q}}    
\def\Quantile{\text{Q}}    
\def\cdf{\text{cdf}}    

\def\indicator{{\mathbf 1}}

\def\I{{\mathbf I}}   % the identity transformation
\def\t{{\mathbf t}}
\def\v{{\mathbf v}}
\def\w{{\mathbf w}}
\def\u{{\mathbf u}}

\def\Fspace{\mathbb{F}}    % a set
\def\Xspace{\mathbb{X}}    % a set
\def\Yspace{\mathbb{Y}}    % a set

\def\Expected{\mathcal{E}}    % expected value

\def\Fset{\mathcal{F}}    % a set
\def\Xset{\mathcal{X}}    % a set
\def\Yset{\mathcal{Y}}    % a set

\def\Aset{\mathcal{A}}    % a set
\def\Bset{\mathcal{B}}    % a set
\def\Rset{\mathcal{R}}    % a set
\def\Sset{\mathcal{S}}    % a set
\def\Tset{\mathcal{T}}    % a set
\def\Xset{\mathcal{X}}    % a set
\def\Trainset{\mathcal{T}_r}    % a set
\def\Testset{\mathcal{T}_e}    % a set
\def\Pset{\mathcal{P}}    % a set
\def\Re{\mathbb{R}}    % Real numbers
\def\Vset{\mathcal{V}}    % a set
\def\Gset{\mathcal{G}}    % a set
\def\Gaussian{\mathcal{G}}    % a set

\def\sign{\mathrm{sign}}    % sign function
\def\support{\mathrm{supp}}    % support of a measure

\def\Pr{\mathrm{Prob}}   % probability

\def\Da#1{{\mathcal{D}{#1}}}    % derivative operator
\def\Db#1#2{{\mathcal{D}{#1}_{\mid_{#2}}}}    % derivative operator
\def\Dc#1#2#3{{\mathcal{D}{#1}_{\mid_{#2}}({#3})}}  % derivative operator
\def\Dd#1#2#3#4{{\mathcal{D}_{#1}{{#2}}_{\mid_{#3}}({{#4}})}} % derivative operator
\def\De#1#2#3{{\mathcal D}_{#1}{#2}_{\mid_{#3}}}  % derivative operator
\def\Df#1#2{{\mathcal D}_{#1}{#2}}  % derivative operator

\def\Ga#1{{\mathbf \nabla}{#1}}   % derivative operator
\def\Gb#1#2{{\mathbf \nabla}{#1}_{\mid_{#2}}} % derivative operator
\def\Gc#1#2#3{{\mathbf \nabla}_{#1}{{#2}}_{\mid_{#3}}}  % derivative operator
\def\Gf#1#2{{\mathbf \nabla}_{#1}{#2}}  % derivative operator

\def\da#1#2{{\partial}_{#1}{#2}}  % partial derivative operator
\def\db#1#2#3{{\partial}_{#1}{#2}_{\mid_{#3}}}  % partial derivative operator

\def\norm#1{{\parallel{#1}\parallel}}   % l2 norm
\def\norm2#1{{\parallel{#1}\parallel^2}}  % squared l2 norm

\def\a{{\mathbf a}} 
\def\l{{\mathbf l}} 
\def\n{{\mathbf n}} 
\def\x{{\mathbf x}} 
\def\p{{\mathbf p}} 
\def\q{{\mathbf q}} 
\def\r{{\mathbf r}} 
\def\f{{\mathbf f}}     % generic vector-valued function
\def\g{{\mathbf g}}     % generic vector-valued function
\def\h{{\mathbf h}}     % generic vector-valued function
\def\e{{\mathbf e}}     % standard basis vectors
\def\y{{\mathbf y}}
\def\z{{\mathbf z}}


\begin{document}
\lstdefinelanguage{clojure}%
{morekeywords={
%Math,Random,List,ArrayList,
deftest,testing,is,defrecord,
*,*1,*2,*3,*agent*,*allow-unresolved-vars*,*assert*,*clojure-version*,*command-line-args*,%
*compile-files*,*compile-path*,*e,*err*,*file*,*flush-on-newline*,*in*,*macro-meta*,%
*math-context*,*ns*,*out*,*print-dup*,*print-length*,*print-level*,*print-meta*,*print-readably*,%
*read-eval*,*source-path*,*use-context-classloader*,*warn-on-reflection*,+,-,->,->>,..,/,:else,%
<,<=,=,==,>,>=,@,accessor,aclone,add-classpath,add-watch,agent,agent-errors,aget,alength,alias,%
all-ns,alter,alter-meta!,alter-var-root,amap,ancestors,and,apply,areduce,array-map,aset,%
aset-boolean,aset-byte,aset-char,aset-double,aset-float,aset-int,aset-long,aset-short,assert,%
assoc,assoc!,assoc-in,associative?,atom,await,await-for,await1,bases,bean,bigdec,bigint,binding,%
bit-and,bit-and-not,bit-clear,bit-flip,bit-not,bit-or,bit-set,bit-shift-left,bit-shift-right,%
bit-test,bit-xor,boolean,boolean-array,booleans,bound-fn,bound-fn*,butlast,byte,byte-array,%
bytes,cast,char,char-array,char-escape-string,char-name-string,char?,chars,chunk,chunk-append,%
chunk-buffer,chunk-cons,chunk-first,chunk-next,chunk-rest,chunked-seq?,class,class?,%
clear-agent-errors,clojure-version,coll?,comment,commute,comp,comparator,compare,compare-and-set!,%
compile,complement,concat,cond,condp,conj,conj!,cons,constantly,construct-proxy,contains?,count,%
counted?,create-ns,create-struct,cycle,dec,decimal?,declare,def,definline,defmacro,defmethod,%
defmulti,defn,defn-,defonce,defprotocol,defstruct,deftype,delay,delay?,deliver,deref,derive,%
descendants,destructure,disj,disj!,dissoc,dissoc!,distinct,distinct?,do,do-template,doall,doc,%
dorun,doseq,dosync,dotimes,doto,double,double-array,doubles,drop,drop-last,drop-while,empty,empty?,%
ensure,enumeration-seq,eval,even?,every?,false,false?,ffirst,file-seq,filter,finally,find,find-doc,%
find-ns,find-var,first,float,float-array,float?,floats,flush,fn,fn?,fnext,for,force,format,future,%
future-call,future-cancel,future-cancelled?,future-done?,future?,gen-class,gen-interface,gensym,%
get,get-in,get-method,get-proxy-class,get-thread-bindings,get-validator,hash,hash-map,hash-set,%
identical?,identity,if,if-let,if-not,ifn?,import,in-ns,inc,init-proxy,instance?,int,int-array,%
integer?,interleave,intern,interpose,into,into-array,ints,io!,isa?,iterate,iterator-seq,juxt,%
key,keys,keyword,keyword?,last,lazy-cat,lazy-seq,let,letfn,line-seq,list,list*,list?,load,load-file,%
load-reader,load-string,loaded-libs,locking,long,long-array,longs,loop,macroexpand,macroexpand-1,%
make-array,make-hierarchy,map,map?,mapcat,max,max-key,memfn,memoize,merge,merge-with,meta,%
method-sig,methods,min,min-key,mod,monitor-enter,monitor-exit,name,namespace,neg?,new,newline,%
next,nfirst,nil,nil?,nnext,not,not-any?,not-empty,not-every?,not=,ns,ns-aliases,ns-imports,%
ns-interns,ns-map,ns-name,ns-publics,ns-refers,ns-resolve,ns-unalias,ns-unmap,nth,nthnext,num,%
number?,odd?,or,parents,partial,partition,pcalls,peek,persistent!,pmap,pop,pop!,pop-thread-bindings,%
pos?,pr,pr-str,prefer-method,prefers,primitives-classnames,print,print-ctor,print-doc,print-dup,%
print-method,print-namespace-doc,print-simple,print-special-doc,print-str,printf,println,println-str,%
prn,prn-str,promise,proxy,proxy-call-with-super,proxy-mappings,proxy-name,proxy-super,%
push-thread-bindings,pvalues,quot,rand,rand-int,range,ratio?,rational?,rationalize,re-find,%
re-groups,re-matcher,re-matches,re-pattern,re-seq,read,read-line,read-string,recur,reduce,ref,%
ref-history-count,ref-max-history,ref-min-history,ref-set,refer,refer-clojure,reify,%
release-pending-sends,rem,remove,remove-method,remove-ns,remove-watch,repeat,repeatedly,%
replace,replicate,require,reset!,reset-meta!,resolve,rest,resultset-seq,reverse,reversible?,%
rseq,rsubseq,second,select-keys,send,send-off,seq,seq?,seque,sequence,sequential?,set,set!,%
set-validator!,set?,short,short-array,shorts,shutdown-agents,slurp,some,sort,sort-by,sorted-map,%
sorted-map-by,sorted-set,sorted-set-by,sorted?,special-form-anchor,special-symbol?,split-at,%
split-with,str,stream?,string?,struct,struct-map,subs,subseq,subvec,supers,swap!,symbol,symbol?,%
sync,syntax-symbol-anchor,take,take-last,take-nth,take-while,test,the-ns,throw,time,to-array,%
to-array-2d,trampoline,transient,tree-seq,true,true?,try,type,unchecked-add,unchecked-dec,%
unchecked-divide,unchecked-inc,unchecked-multiply,unchecked-negate,unchecked-remainder,%
unchecked-subtract,underive,unquote,unquote-splicing,update-in,update-proxy,use,val,vals,%
var,var-get,var-set,var?,vary-meta,vec,vector,vector?,when,when-first,when-let,when-not,%
while,with-bindings,with-bindings*,with-in-str,with-loading-context,with-local-vars,%
with-meta,with-open,with-out-str,with-precision,xml-seq,zero?,zipmap
},%
   sensitive,% ???
   alsodigit=-,%
   morecomment=[l];,%
   morestring=[b]"%
  }[keywords,comments,strings]%
 


\title{Regression costs for decision trees}


\author{\textsc{John Alan McDonald }}


\date{\today}
\maketitle

The purpose of this document is work thru an alternative to $L_2$ cost
that is a bit more efficient to compute, and gives the same results when
choosing split predicates in decision tree growing.

\section{\label{sub:Decision-trees}Greedy decision trees}

A general binary decision tree consists of 
\begin{itemize}
\item internal \emph{split} nodes, each containing a predicate that determines
whether a record goes to the left or right child of that node.
\item terminal \emph{leaf} nodes, each containing a leaf model function
whose value is the tree's prediction for any record that ends up in
that node. 
\end{itemize}

Greedy split optimization --- choose the best out of all feasible
splits, and repeat on the resulting child nodes until there are no
feasible splits --- is the most common way of growing decision trees.
It depends on several things:
\begin{enumerate}
  \item A cost function $c$ used to define 'best'.  
  \item An enumeration of splits to consider. Pure greedy splitting
  considers all 'feasible' splits on all attributes.
  \begin{enumerate} 
    \item For categorical attributes, that, in general, means considering every 
    partition of the categories into $2$ subsets. However, for some
    important cost functions (eg Gini, $L_2$), it can be shown that the
    optimal split can be found by sorting the categories by the
    corresponding score function (eg the response mean for $L_2$ cost),
    and then considering only splits by score.
    \item For numerical attributes, the most general split would come
    from treating the distinct values like the categories of a
    categorical variable. However, no one does that, mostly because
    there are usually too many distinct values. Instead, only splits by
    $\leq$ vs $>$ one of the distinct values are considered.
\end{enumerate}
\item A feasibility test that determines whether a given split on a
given attribute is allowed. The most common case here is to require both
children of the split contain some minimum number of training records.
\end{enumerate}

\section{\label{sec:numerical}Cost functions for $L_2$ numerical
regression}

Let $\mathcal{T} = \{ \left( y,\mathbf{x} \right) \}$ be the training
data in the node to be split.
It is a set of pairs of predictor record $\mathbf{x}$ and ground truth
response $y$, where $y\in\mathbb{R}$ for numerical regression.
We are considering splits on some particular predictor field $x_k$,
which might be numerical or categorical.

The cost function for $L_2$ regression is the sum of squared deviations
from the mean: 

$L_{2}\left(\mathcal{T}\right)
= \sum_{y\in\mathcal{T}}\,\left(y-\bar{y}_{\mathcal{T}}\right)^{2}$,
where $\bar{y}_{\mathcal{T}}
= \frac{1}{\#\mathcal{T}}\sum_{y\in\mathcal{T}}\,y$.

Note that computing this \textit{accurately}, in an online fashion, for
moderate $\#\mathcal{T}$, the number of records in $\mathcal{T}$, allowing for
the updating/downdating needed for fast split optimization, 
requires some care. 

However, a little bit of algebra will let us use a simpler alternative
to get the same splits. 


Any split partitions the training y-values
$\mathcal{T=}\left\{ y\right\} $ into left and right subsets: 
$\mathcal{T}=\mathcal{L}\uplus\mathcal{R}$.
The split cost is:

\begin{align*}
c\left(\mathcal{L},\mathcal{R}\right)= & L_{2}\left(\mathcal{L}\right)
+ L_{2}\left(\mathcal{R}\right)\\
= & \sum_{y\in\mathcal{L}}\,\left(y-\bar{y}_{\mathcal{L}}\right)^{2}+\sum_{y\in\mathcal{R}}\,\left(y-\bar{y}_{\mathcal{R}}\right)^{2}\\
= & \sum_{y\in\mathcal{L}}\left[y^{2}-2\bar{y}_{\mathcal{L}}y+\bar{y}_{\mathcal{L}}^{2}\right]+\sum_{y\in\mathcal{R}}\left[y^{2}-2\bar{y}_{\mathcal{R}}y+\bar{y}_{\mathcal{R}}^{2}\right]\\
= & \sum_{\mathcal{L}\uplus\mathcal{R}}y^{2}-\frac{\left(\sum_{\mathcal{L}}y\right)^{2}}{\#\mathcal{L}}-\frac{\left(\sum_{\mathcal{R}}y\right)^{2}}{\#\mathcal{R}}
\end{align*}

Since $\sum_{\mathcal{L}\uplus\mathcal{R}}y^{2}$ doesn't depend on
the split, minimizing $c\left(\mathcal{L},\mathcal{R}\right)$ is
equivalent to minimizing $-\left[\frac{\left(\sum_{\mathcal{L}}y\right)^{2}}{\#\mathcal{L}}+\frac{\left(\sum_{\mathcal{R}}y\right)^{2}}{\#\mathcal{R}}\right]$,
so we can use $\frac{-\left(\sum_{\mathcal{T}}y\right)^{2}}{\#\mathcal{T}}$
as our cost function in split optimization.

\section{Cost functions for $L_2$ vector-valued regression}

Let $\mathcal{T} = \{ \left( \mathbf{y}, \mathbf{x} \right) \}$ be the
training data in the node to be split.
Here the ground truth response $\mathbf{y}$ is a vector, 
$\mathbf{y}\in\mathbb{R}^m$, rather than a single number.

The cost function for $L_2$ vector-valued regression is the sum of
squared $L_2$ distances from the mean vector: 
\begin{align*}
L_{2}\left(\mathcal{T}\right) 
= &
\sum_{y\in\mathcal{T}}\,\|\mathbf{y}-\bar{\mathbf{y}}_{\mathcal{T}}\|_{2}^{2}\\
= &
\sum_{y\in\mathcal{T}}\sum_{i=0}^{m-1}\,\left(y_i-\bar{y_i}_{\mathcal{T}}\right)^{2}
\end{align*}

Following the same reasoning as in section~\ref{sec:numerical}, we get
for a simpler cost:
$$
c\left(\mathcal{L},\mathcal{R}\right) = 
-\left[
\frac{
\sum_{i=0}^{m-1} \, \left(\sum_{\mathcal{L}}y_i\right)^{2}}
{\#\mathcal{L}}
\; + \;
\frac{\sum_{i=0}^{m-1} \,
\left(\sum_{\mathcal{R}}y_i\right)^{2}}{\#\mathcal{R}}\right] 
$$

%\bibliographystyle{plainurl}
%\bibliography{bib,cactus}

\end{document}
